\documentclass[sigconf]{acmart}

\settopmatter{printacmref=false} % Removes citation information below abstract
\renewcommand\footnotetextcopyrightpermission[1]{} % removes footnote with conference information in first column
\pagestyle{plain} % removes running headers

\usepackage{booktabs} % For formal tables


% Copyright
%\setcopyright{none}
%\setcopyright{acmcopyright}
%\setcopyright{acmlicensed}
\setcopyright{rightsretained}
%\setcopyright{usgov}
%\setcopyright{usgovmixed}
%\setcopyright{cagov}
%\setcopyright{cagovmixed}
\usepackage{graphicx}

\copyrightyear{2017}


\acmArticle{4}
\acmPrice{15.00}


\begin{document}
\title{A Chance to Work}
\subtitle{Understanding the composition of foreign workers 
pursuing specialty occupations on the United States H1-B visa
}


\author{\textbf{Alexander Buddenbaum}}
\affiliation{%
  \institution{Georgia Institute of Technology}
%   \streetaddress{North Avenue}
  \city{Shenzhen} 
  \country{China}
}
\email{alex.budd@gatech.edu}

\author{\textbf{Qinrui Li}}
\affiliation{%
  \institution{Georgia Institute of Technology}
%   \streetaddress{North Avenue}
  \city{Shenzhen} 
  \country{China} 
}
\email{qli449@gatech.edu}

\author{\textbf{Tianyu Li}}
\affiliation{%
  \institution{Georgia Institute of Technology}
%   \streetaddress{North Avenue}
  \city{Shenzhen} 
  \country{China}  
}
\email{tli303@gatech.edu}


\author{\textbf{Chuanqi Liu}}
\affiliation{%
  \institution{Georgia Institute of Technology}
%   \streetaddress{North Avenue}
  \city{Shenzhen} 
  \country{China}
}
\email{cliu732@gatech.edu}

\author{\textbf{Tianshu Tao}}
\affiliation{%
  \institution{Georgia Institute of Technology}
%   \streetaddress{North Avenue}
  \city{Shenzhen} 
  \country{China} 
}
\email{ttao35@gatech.edu}



\maketitle

\section{Introduction}

placeholder

\section{Related work}

placeholder

\section{Proposed method}
In order to show comprehensive aspects of H1-B statistics to the potential users, we propose a method of 
heterogeneous visualization method employing multiply widely used charts implemented by popular Vue and Flask.

Users will select the h1-b information which they are interested in on our portal page. Their request will be 
passed to our backend REST API that was implemented by Python Flask. The API will then send the corresponding 
statistics back to the front-end interface. The UI will then process the data and render the chart on the newly-directed 
page. And the user can view the chart and use the mouse to interact with the chart. The front-end application 
based on Vue and D3 is expected to be totally interactive so as to provide the user rich choices of viewing, 
comparing, selecting and even searching by certain fields.

There must be some arguments that this could all be done by h1bgrader.com. However, we are introducing three main 
innovations. To begin with, our system will provide users more selections and interaction of the chart types. At least 
cases by state choropleth chart has never been implemented by that known webstie. Besides, we would offer more 
significant information like search by section and search by state that the users must have to visit other websites for. Last but not 
least, our project introduced the machine learning method to predict the visa statistical trend which these kind of 
websites never provide.

All the h1-b case data was retrieved from DOL official website, which makes our source authoritative and trustworthy. 
However, the naming rubric change still happened through these years. Therefore, OpenRefine is used to conformalize the data 
extracting the columns the project actually needs.

The cost of the development is terrifically economic. All of our data and framework are free, and we will be hosting the system 
on any local business machine.

\section{Experiments and evaluation}
placeholder

\section{Conclusion}

placeholder
\appendix
\section{Timelines}
\subsection{Current timeline}

\begin{itemize}
	\item November 13: Database, backend, and frontend connected and communicating
	\item November 20: Application functioning on small dataset
	\item November 24: Poster design complete
	\item November 27: Application polished and presentable for poster presentations
	\item November 30: Poster presentation, full dataset uploaded
	\item December 4: Final report complete
\end{itemize}

\subsection{Previous timeline}
\begin{itemize}
	\item October 23: Able to query locations from front-end, all data added to database
	\item November 6: Simple path generated and displayed based on user-provided start and end points
	\item November 10: Progress report due
	\item November 20: User can select options to impact path generation, continue to refine algorithm and front-end
	\item November 28: Poster presentation
	\item December 5: Project complete, final report due
\end{itemize}

\section{Work distribution}

\begin{itemize}
	\item Scraping and collecting data, geocaching (Alex)
	\item Cleaning and standardizing data (Chuanqi)
	\item Designing our algorithm (Qinrui)
	\item Implementing the backend (Tianshu)
	\item Creating the front-end (Tianyu)
\end{itemize}

All team members have contributed similar amount of effort.

\section{Innovations}
\begin{itemize}
	\item Incorporating trail quality data
	\item Using Tarjan's algorithm to generate paths
	\item Scraping and providing points of interest specifically geared towards bikers and hikers
\end{itemize}

Further possible innovations we would like to implement if we have time:
\begin{itemize}
	\item Hikers can update trails and give feedback
	\item One can edit the trail he wants to follow directly in the interface (by drag and drop for example)
	\item One can export the trail to a gps thanks to a gpx file
\end{itemize}


\begin{thebibliography}{9}
\bibitem{} 
Agarwal, Sathak and KS Rajan. 
\textit{Analyzing the performance of NoSQL vs. SQL databases for Spatial and Aggregate queries.}. 
Free and Open Source Software for Geospatial Conference Proceedings 17, 2017.
 
\bibitem{} 
Allahbakhsh, Mohammad, Boualem Benatallah, and Aleksandar Ignjatovic.
\textit{Quality Control in Crowdsourcing Systems.}.
Web-Scale Workflow, 76-81, 2013.

\bibitem{} 
Amirian, Pouria, Anahid Basiri, and Adam Winstanley.
\textit{Evaluation of data management systems for geospatial big data}.
International Conference on Computational Science and Its Applications. Springer, Cham, 2014.

\bibitem{} 
Dijkstra, Edsger W.
\textit{A note on two problems in connexion with graphs}.
Numerische mathematik 1.1 (1959): 269-271.

\bibitem{} 
Fielding, Roy Thomas.
\textit{Architectural Styles and the Design of Network-based Software Architectures}.
Doctoral Dissertation, University of California, Irvine, 2000.

\bibitem{} 
Johnson, Donald B.
\textit{Finding all the elementary circuits of a directed graph}.
SIAM Journal on Computing 4.1 (1975): 77-84.

\bibitem{} 
Haklay, Mordechai.
\textit{How good is volunteered geographical information? A comparative study of OpenStreetMap and Ordnance Survey datasets}.
Environment and Planning B: Planning and Design 37 (2010): 682-703.

\bibitem{} 
Holte, Robert C.
\textit{Very Simple Classification Rules Perform Well on Most Commonly Used Datasets}.
Computer Science Department, University of Ottawa, 1993.

\bibitem{} 
Provost, Foster, and Tom Fawcett.
\textit{Chapter 7: Decision Analytic Thinking I: What Is a Good Model?}
Data Science for Business: What You Need to Know about Data Mining and Data-Analytic Thinking, O'Reilly (2013).

\bibitem{} 
Samer Buna.
\textit{All the fundamental React.js concepts}
Medium (blog), Aug 2017.
\\\texttt{goo.gl/LUMxtp}

\bibitem{} 
Provost, Foster, and Tom Fawcett.
\textit{Chapter 8: Visualizing Model Performance?}
Data Science for Business: What You Need to Know about Data Mining and Data-Analytic Thinking, O'Reilly (2013).

\bibitem{} 
Rosenzweig, Elizabeth.
\textit{Successful User Experience: Strategies and Roadmaps}.
Elsevier Science (2015): Chapter 3.

\bibitem{} 
Colt, Pini.
\textit{React + Redux: Architecture Overview}
Medium, Nov 2016.
\\\texttt{goo.gl/8xYN6Q}

\bibitem{} 
Tarjan, Robert.
\textit{Depth-first search and linear graph algorithms}.
SIAM journal on computing 1.2 (1972): 146-160.

\bibitem{} 
Paris, Michel.
\textit{REST 2.0 is here and it’s name is GraphQL}
SitePoint (blog), May 17, 2017.
\\\texttt{https://www.sitepoint.com/rest-2-0-graphql/}


\end{thebibliography}

\end{document}
